
\chapter{HABILIDADES SOCIOEMOCIONALES}

\parbox{4cm}{ \includegraphics[scale=0.38]{unsch}} \parbox{10.8cm}{ Son las habilidades para dominar y controlar las emociones propias y de los demás y orientar con ello el pensamiento y las acciones(Lorenzo,2006; Rodriguez,2013 y Riquelme,2013). La competencia emocional le va a dotar de herramientas y estrategias emocionales que le permitirán afrontar de forma coherente y digna su vida cotidiana (Riquelme,2013).}%494,6,6}\\
\\
Las investigaciones que sobresalen son los propuestos por: Goleman que tiene como eje a las organizaciones ,Bar-On a los aspectos sociales y Salovey/Mayer a las habilidades cognitivas (Valdivia, 2006).%1020154558


\section{MAYER Y SALOVEY}

%==================================================================0
\\\\Proponen un modelo que agrupa las habilidades desde lo más básico a lo más complejo en cuanto al manejo emocional(Valdivia, 2006, p.43)\\
Los siguientes 4 niveles son un resumen de dicho modelo:
\begin{enumerate}[label=\itembolasazules{\arabic*}]
\item \textbf{Percepción y Expresión Emocional:}\\
 Es la habilidad más simple que permite percibir y expresar las emociones sentimientos o pensamientos propios y de los demás, en nuestro estado físico mediante conductas, lo verbal y creaciones artísticas. También esta habilidad es necesaria para discriminar lo adecuado o no, lo honesta o no de las expresiones emocionales.
\item \textbf{Asimilación Emocional por parte del Pensamiento:}\\
 Habilidad para
describir eventos emocionales que asisten al proceso intelectual y que genera a la vez emociones resultantes del proceso mencionado. Un ejemplo de este proceso es el siguiente Estado de Animo positivo - pensamiento optimista; estado de animo negativo - pensamiento pesimista. También esta habilidad ayuda a generar múltiples perspectivas permitiendo a la emoción el ser un asistente para los actos creativos y la solución de problemas.
\item \textbf{Comprensión de las Emociones:}\\
 Habilidad para categorízar a las
emociones, para entender su significado y el de los sentimientos compiejos o simultáneos. También ayuda a la transición de un estado emocional a otro sin exagerar o minimizar la importancia de las emociones.
\item \textbf{Regulación Reflexiva de las Emociones:}\\
 Es la habilidad consciente del individuo para regular las emociones en si mismo y en los otros, para intensificar el crecimiento intelectual y emocional, para tolerar lo placentero o displacentero de las emociones/ Esta habilidad promueve la atención y apertura en los sentimientos para aprender de ellos.
\end{enumerate}







%====================================================================

\section{BAR-ON}
\\\\Para Bar-On las habilidades socio-emocionales cambian con el tiempo, y están asociadas con el bienestar personal.\\
Bar-On(2005, citado en Valdivia, 2006, p.182) define las habilidades socioemocionales como:
\begin{caja}[]
{\it una intersección entre competencias emocionales y sociales, las cuales son habilidades y facilitadores que determinan que tan efectivamente nos comprendemos y expresamos a nosotros mismos, comprendemos a los demás y nos relacionarnos con ellos, asi como lidiamos con las demandas diarias.}%1020154558
\end{caja}\\\\
%=================================================================
%==================================================================0
%(1020154558)VALDIVIA J.(2006, P.30)
Para Bar-On el ser humano posee las siguientes habilidades o competencias socio - emocionales:\\
\begin{enumerate}[label=\itembolas{}]
\item Autoconciencia y auto expresión
\item Conciencia social y relaciones interpersonales
\item Manejo de Stress y Emociones
\item Adaptabilidad y Manejo de Cambios
\item Estado General de Automotivación
\end{enumerate}

Bar-On integra cada una de las habilidades mencionadas en un modelo al que denomina la Inteligencia Social-Emocional (ESI).\\\\

%=================================================================
%==================================================================0
Para medir estas habilidades , Bar-On ha desarrollado el Emocional Quotient Inventory (EQ-i)(Valdivia,2006), instrumento en forma de autoreporte cuyas 5 categorías contienen las siguientes 15 habilidades:\\

\begin{enumerate}[label=\itembolasazules{}]
\item Intrapersonal: \\
Autoconsideración, Autoconciencia Emocional, Asertividad, Independencia y Autoactualización.
\item Interpersonal: \\
Empatia, Responsabilidad Social, Relaciones Interpersonales.
\item Manejo de Stress: \\
Tolerancia al Stress y Control de Impulsos.
\item Adaptabilidad: \\
Evaluación de la realidad, Flexibilidad, Solución de problemas.
\item Estado General de Animo: \\
Optimismo, Felicidad.
\end{enumerate}

%================================================================












\section{DANIEL GOLEMAN}



\begin{caja}[]
"La competencia emocional para Goleman es una capacidad basada en Inteligencia Emocional que toma en cuenta las habilidades personales y sociales necesarias para un desempeño laborar sobresaliente"(Chemiss, 2000, citado en Valdivia, 2006, p.36)%1020154558
\end{caja}\\\\


Desde este modelo, Goleman (2004,citado en Valdivia, 2006, p.37), propone que:
\begin{definicion}[]
{\it ... Las competencias necesarias para el liderazgo pueden ser aprendidas y además se mantienen con el tiempo. Asimismo, es importante que el líder este motivado... y se esfuerce en llevar a afecto la tarea siendo consciente de que puede hacerlo solo recibe la retroalimentación idónea, la guía y el soporte de otras personas. Por tal razón, es necesario que el líder "desaprenda hábitos en la forma de pensar, sentir y actuar, que pueden estar profundamente arraigados, para desarrollar los nuevos."}
\end{definicion}
\\%51020154558


Segun el autor considera que la habilidad técnica del líder le permite conocer su tarea, pero es su competencia emocional la que impulsa el trabajo del equipo y la consecución de los logros fijados, asi el autor menciona las siguientes competencias necesarias en el líder(Valdivia, 2006, Lorenzo, 2006 y Tejedor, 2013):
\\\\%51020154558,494,6

\begin{enumerate}[label=\itembolasazules{\arabic*}]
\item Autoconciencia: \\
\begin{figure}[!h]
		% 60% de la página
		\begin{minipage}[b]{0.15\textwidth}
			\hspace{1cm} {\includegraphics[scale=0.1]{unsch}}%(1020154558,6)
			% 30% de la pág
		\end{minipage} \hfill \begin{minipage}[b]{0.83\textwidth}
		% Figuras: ver capítulo 5
			Habilidad para la autoe valuación sincera y realista.Nos referimos a la capacidad de saber lo que se siente en un determinado momento. Una capacidad básica para guiar la toma de decisiones y no quedar a merced de las emociones incontroladas.
	\end{minipage}
\end{figure}

\item Autorregulación: \\
\begin{figure}[!h]
		% 60% de la página
		\begin{minipage}[b]{0.25\textwidth}
			\hspace{1cm} {\includegraphics[scale=0.099]{unsch}}%(1020154558,6)
			% 30% de la pág
		\end{minipage} \hfill \begin{minipage}[b]{0.73\textwidth}
		% Figuras: ver capítulo 5
			Habilidad para detectar los impulsos o emociones negativas y encausarlas hacia formas productivas.Su habilidad lleva consigo el manejo adecuado de expresiones de ira, furia o irritabilidad, tan fundamental en las relaciones interpersonales.
	\end{minipage}
\end{figure}
\item Motivación: 
\begin{figure}[!h]
		% 60% de la página
		\begin{minipage}[b]{0.25\textwidth}
			\hspace{1cm} {\includegraphics[scale=0.099]{unsch}}%(1020154558,6)
			% 30% de la pág
		\end{minipage} \hfill \begin{minipage}[b]{0.73\textwidth}
		% Figuras: ver capítulo 5
			Habilidad para mantenerse motivado hacia la tarea por el placer de realizaría que combinada con la autorregulación permite superar las frustraciones.
	\end{minipage}
\end{figure}
\item Empatia: 
\begin{figure}[!h]
		% 60% de la página
		\begin{minipage}[b]{0.15\textwidth}
			\hspace{1cm} {\includegraphics[scale=0.1]{unsch}}%(1020154558,6)
			% 30% de la pág
		\end{minipage} \hfill \begin{minipage}[b]{0.83\textwidth}
		% Figuras: ver capítulo 5
			Habilidad para evaluar a conciencia los sentimientos de los demás mientras se realizan decisiones que afectan al equipo de trabajo. Las personas empáticas sintonizan mejor con las necesidades de los demás.
	\end{minipage}
\end{figure}
\item Habilidad Social:
\begin{figure}[!h]
		% 60% de la página
		\begin{minipage}[b]{0.15\textwidth}
			\hspace{1cm} {\includegraphics[scale=0.1]{unsch}}%(1020154558,6)
			% 30% de la pág
		\end{minipage} \hfill \begin{minipage}[b]{0.83\textwidth}
		% Figuras: ver capítulo 5
			Habilidad de una persona para manejar las relaciones con los otros, al comprender y controlar las emociones propias y de los otros.La competencia social y las habilidades que conlleva ( interactuar fluidamente, persuadir, dirigir, negociar, y resolver disputas) constituyen la base del liderazgo y la eficacia.
	\end{minipage}
\end{figure} 
\end{enumerate}

Estas mismas dimensiones se destacan en otros autores, como la profesora Caballero (Rodriguez, 2013, p.182)
\\

Goleman(2000, citado en Valdivia, 2006) menciona:
\begin{definicion}[]
{ \it Los líderes efectivos al utilizar estas competencias emocionales son sensibles de las situaciones que se les presentan, lo cual les permite la flexibilidad para realizar los cambios pertinentes en su liderazgo. Por contrario, los dirigentes que carecen de estas competencias no logran alcanzar las metas propuestas y modifican negativamente el ambiente laboral con sus subalternos.}
\end{definicion}\\\\


%=================================================================
Goleman junto a Boyatzis plantean 5 dimensiones que integran 25 competencias(Valdivia, 2006, p.36):%1020154558

\begin{enumerate}[label=\itembolasazules{\arabic*}]
\item \textbf{Autoconciencia:} \\
Conciencia Emocional, Autoevaluación adecuada, autoconfianza.
\item \textbf{ Autorregulación:}\\
 Autocontrol, Confiabilidad, Consciencia, Adaptabilidad e Innovación.
\item \textbf{Motivación:} \\
Impulso para el desempeño, Compromiso, iniciativa y optimismo
\item \textbf{Empatia:} \\
Comprensión hacia los otros, Desarrollo para los otros, Orientación de Servicio, Impulso a diversidad y Conciencia Política.
\item \textbf{Habilidades Sociales:} \\
Influencia, Comunicación, Manejo de Conflictos, Liderazgo, Catalizador para el cambio, Construcción de Alianzas, Colaboración y cooperación y Capacidades para el trabajo en equipo.
\end{enumerate}

Para su medición, los autores desarrollaron el Emotional Competence Inventory (ECI) basándose en un instrumento anterior: el Self-Assessment Questionnaire de Boyatsis (Cherniss, 2000, citado en Valdivia, 2006).
%=======================================================================
\\\\

%==================================================================0
%(6
%Boyatzis, Goleman y Rhee(2000, citado en Riquelme, 2013, p.278) nos %menciona "veinte competencias emocionales sobre la base de cuatro %dimensiones  que se combinan entre sí: reconocimiento, regulación, %competencia personal (uno mismo) y competencia social (en los %demás)."

%HAY FIGURAS VER EN LIBRO SACAR DE AHI.
%=================================================================


